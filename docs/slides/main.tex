\documentclass[xcolor=table, aspectratio=169]{beamer}
\usepackage{graphicx}

% --- Core French Language Support ---
\usepackage[T1]{fontenc}
\usepackage[utf8]{inputenc}
\usepackage[french]{babel}
\usepackage{listings}
\usepackage{booktabs} 
\usepackage{tikz} 
\usetikzlibrary{shapes,arrows,positioning}
\usetheme{Darmstadt}
\usecolortheme{dolphin}
\usefonttheme{professionalfonts}

\usepackage{csquotes}

\AtBeginSection[]{
\begin{frame}
    \frametitle{Sommaire}
    \tableofcontents[currentsection]
\end{frame}
}

%% --- Metadata ---
\title[DistCapsule]{DistCapsule : Distributeur Connecté}
\subtitle{Système IoT Biométrique sur Raspberry Pi 5}
\author[Z. Wang \& X. Tang]{Zhaoyu WANG et Xinqi TANG}
\date[\today]{Soutenance Finale -- \today}
\institute[M2 MIASHS]{Université Paris 8 \\ Master Technologies et Handicaps}

%% --- Logo Configuration ---
\titlegraphic{
   \includegraphics[height=1.2cm]{pic/logo-handi.png} \hfill
   \includegraphics[height=1.2cm]{pic/logo-p8.png}
}
\logo{\includegraphics[width=0.8cm]{pic/logo-p8.png}}

%% --- Code Snippet Style ---
\lstset{
    basicstyle=\ttfamily\scriptsize,
    keywordstyle=\color{blue!80!black}\bfseries,
    stringstyle=\color{green!60!black},
    commentstyle=\color{gray}\itshape,
    numbers=left,
    numberstyle=\tiny\color{gray},
    frame=shadowbox,
    rulesepcolor=\color{gray},
    breaklines=true,
    language=Python,
    showstringspaces=false
}

\begin{document}

{
\logo{}
\begin{frame}
    \titlepage   
\end{frame}
}

% --- Global Table of Contents ---
\begin{frame}{Plan de la présentation}
    \tableofcontents
\end{frame}

% ==============================================================================
% SECTION 1: CONTEXTE (3-4 min)
% ==============================================================================
\section{Contexte et Problématique}

\begin{frame}{Le Constat}
    Dans un espace café partagé, les capsules sont souvent mélangées en vrac ou stockées individuellement, encombrant inutilement le plan de travail.
    
    \begin{columns}
        \begin{column}{0.6\textwidth}
            \begin{alertblock}{Problèmes Identifiés}
                \begin{itemize}
                    \item \textbf{Désordre} : Capsules en vrac, boîtes empilées, perte d'espace sur le plan de travail.
                    \item \textbf{Confusion} : Difficulté à distinguer ses propres capsules (variétés similaires).
                    \item \textbf{Stockage} : Certains utilisateurs cachent leurs capsules pour éviter le vol ou le mélange.
                \end{itemize}
            \end{alertblock}
        \end{column}
        \begin{column}{0.38\textwidth}
            \begin{figure}
                \includegraphics[width=\textwidth,height=4cm,keepaspectratio]{pic/espace_cafe.png}
                \caption{Situation actuelle (Stockage passif)}
            \end{figure}
        \end{column}
    \end{columns}
\end{frame}

\begin{frame}{Notre Solution : DistCapsule}
    Transformer un simple présentoir en un \textbf{système embarqué} intelligent.
    
    \vspace{0.3cm}
    \begin{columns}
        \begin{column}{0.6\textwidth}
            \begin{block}{Proposition de Valeur}
                \begin{itemize}
                    \item \textbf{Centralisation} : Une tour compacte stockant 5 variétés.
                    \item \textbf{Personnalisation} : Chaque utilisateur a ses droits d'accès.
                    \item \textbf{Fluidité} : Distribution Biométrique ou via Mobile.
                \end{itemize}
            \end{block}
        \end{column}
        \begin{column}{0.35\textwidth}
            \centering
            \includegraphics[width=\textwidth,height=4cm,keepaspectratio]{pic/amazon_ref.jpg}
            \par\vspace{0.2cm}\small Présentoir à transformer
        \end{column}
    \end{columns}
\end{frame}

% ==============================================================================
% SECTION 2: ARCHITECTURE (5 min)
% ==============================================================================
\section{Architecture Technique}

\begin{frame}{Architecture AAA (3-Tiers)}
    Pour garantir la stabilité, nous avons découplé les responsabilités en trois couches distinctes :
    
    \vspace{0.3cm}
    \begin{center}
    \begin{tikzpicture}[node distance=2.5cm, auto, thick]
        % Nodes - AAA Convention
        \node[draw, fill=green!20, minimum size=1cm, align=center] (Android) {\textbf{Android}\\(Client)};
        \node[draw, fill=blue!20, minimum size=1cm, align=center, right of=Android, node distance=5cm] (Agent) {\textbf{Agent Artificiel}\\(FastAPI)};
        \node[draw, fill=red!20, minimum size=1cm, align=center, right of=Agent, node distance=5cm] (Arduino) {\textbf{Arduino}\\(Raspberry Pi 5)};
        
        % Arrows - Simplified labels
        \draw[<->] (Android) -- node[above] {REST API} (Agent);
        \draw[<->] (Agent) -- node[above] {Commands} (Arduino);
    \end{tikzpicture}
    \end{center}
    
    \begin{itemize}
        \item \textbf{Android} : Interface utilisateur mobile (Java).
        \item \textbf{Agent} : Passerelle de sécurité et orchestration (Python).
        \item \textbf{Arduino} : Contrôle matériel temps réel (lgpio, Servos, Capteurs).
    \end{itemize}
\end{frame}

\begin{frame}{Flux de Données (Workflow)}
    Exemple : Un utilisateur demande une capsule via l'application.
    \vspace{0.2cm}
    \begin{columns}[T]
    \begin{column}{0.35\textwidth}    
    \begin{enumerate}
        \item \textbf{Requête} : \par\texttt{POST} avec Token.
        \item \textbf{Validation} : \par Auth \& Insertion BDD.
        \item \textbf{Exécution} : \par \textbf{Core Agent} (Polling).
        \item \textbf{Action} : \par Pilotage Servo (\texttt{lgpio}).
        \item \textbf{Feedback} : \par Feedback IPS \& App.
    \end{enumerate}
    \end{column}
    \begin{column}{0.65\textwidth}
    \begin{figure}
        \centering
        \includegraphics[width=0.9\textwidth,height=5cm,keepaspectratio]{pic/data_flow.png}
        \par\vspace{0.2cm}\small Schéma du flux de données entre les composants
    \end{figure}

    \end{column}
    \end{columns}
\end{frame}

\begin{frame}{Réseau : Topologie et Configuration}
    \begin{center}
    \begin{tikzpicture}[node distance=2.5cm, auto, thick]
        % Nodes
        \node[draw, fill=green!20, minimum size=0.8cm, align=center] (Phone) {\textbf{Android}\\App};
        \node[draw, fill=red!20, minimum width=2.5cm, minimum height=1.2cm, align=center, right of=Phone, node distance=5cm] (Pi) {\textbf{Raspberry Pi 5}\\wlan0 (Hotspot) + eth0};
        \node[draw, fill=blue!20, minimum size=0.8cm, align=center, right of=Pi, node distance=5cm] (PC) {\textbf{PC Admin}\\Internet + SSH};
        
        % Arrows
        \draw[<->] (Phone) -- node[above] {\scriptsize Wi-Fi} node[below] {\scriptsize 192.168.4.x} (Pi);
        \draw[<->] (Pi) -- node[above] {\scriptsize RJ45} node[below] {\scriptsize 192.168.3.x} (PC);
    \end{tikzpicture}
    \end{center}
    
    \vspace{0.2cm}
    \begin{columns}
        \begin{column}{0.48\textwidth}
            \textbf{Interface Hotspot (wlan0) :}
            \begin{itemize}
                \item \textbf{SSID} : \texttt{DistCapsule\_Box}
                \item \textbf{IP} : \texttt{192.168.4.1}
                \item \textbf{Port API} : \texttt{8000}
                \item \textbf{Sécurité} : WPA2 + Token
            \end{itemize}
        \end{column}
        \begin{column}{0.48\textwidth}
            \textbf{Interface Ethernet (eth0) :}
            \begin{itemize}
                \item \textbf{IP} : \texttt{192.168.3.14}
                \item \textbf{Usage} : SSH, Internet (Git, Pip install, Mises à jour)
                \item \textbf{Internet} : Accès via PC (ICS)
            \end{itemize}
        \end{column}
    \end{columns}
\end{frame}

% ==============================================================================
% SECTION 3: ARDUINO (Matériel)
% ==============================================================================
\section{Arduino (Raspberry Pi 5 - Matériel)}

\begin{frame}{Raspberry Pi 5 : Composants Intégrés}
    \begin{columns}
        \begin{column}{0.57\textwidth}
            \textbf{Périphériques Connectés :}
            \begin{itemize}
                \item \textbf{5x Servos SG90} : PWM logiciel, GPIO 18/6/12/13/19.
                \item \textbf{DY-50} : Empreintes digitales (UART 57600).
                \item \textbf{Camera Module 3} : Reconnaissance faciale (CSI, IMX708).
                \item \textbf{ST7789 LCD} : Écran 240×240 (SPI).
                \item \textbf{Bouton Wake-Up} : Réveil système (GPIO 26).
            \end{itemize}
            
            \begin{block}{Alimentation}
                Pi 5 : USB-C 27W \par Servos : 5V externe (masse commune)
            \end{block}
        \end{column}
        \begin{column}{0.4\textwidth}
            \centering
            \includegraphics[width=\textwidth,height=4cm,keepaspectratio]{pic/hardware_wiring.jpg}
            \par\vspace{0.2cm}\small Câblage réel du prototype
        \end{column}
    \end{columns}
\end{frame}

\begin{frame}{Défi Hardware : La Révolution Pi 5}
    Le passage au Raspberry Pi 5 a cassé la compatibilité de nombreuses bibliothèques historiques.
    
    \begin{itemize}
        \item \textbf{Obsolescence} : \texttt{RPi.GPIO} ne fonctionne pas sur la puce RP1 du Pi 5.
        \item \textbf{Solution} : Migration vers \textbf{\texttt{lgpio}} (Linux GPIO).
    \end{itemize}
    
    \vspace{0.3cm}
    \begin{block}{Innovation : Soft-PWM}
        Pi 5 : 4 canaux HW-PWM (1 réservé ventilateur) $\rightarrow$ Soft-PWM pour 5 servos.
    \end{block}
\end{frame}

\begin{frame}{Câblage et Intégration}
    \begin{center}
    \begin{tikzpicture}[node distance=1cm, auto, thick, scale=0.7, every node/.style={scale=0.7}]
        % Central Pi
        \node[draw, fill=red!20, minimum width=2cm, minimum height=1cm, align=center] (Pi) {\textbf{Pi 5}\\GPIO};
        
        % Direct connections (no breadboard)
        \node[draw, fill=blue!20, minimum width=1.2cm, align=center, above left=0.8cm and 1.2cm of Pi] (LCD) {\textbf{LCD}\\SPI};
        \node[draw, fill=purple!20, minimum width=1.2cm, align=center, above right=0.8cm and 1.2cm of Pi] (Cam) {\textbf{Cam}\\CSI};
        
        % Breadboard (soldered)
        \node[draw, fill=yellow!40, minimum width=3cm, minimum height=0.6cm, align=center, below=1.2cm of Pi] (Bread) {\textbf{Breadboard} (soudé)};
        
        % Through breadboard
        \node[draw, fill=orange!20, minimum width=1.5cm, align=center, below left=0.6cm and 0.3cm of Bread] (Servos) {\textbf{5x SG90}};
        \node[draw, fill=green!20, minimum width=1.2cm, align=center, below=0.6cm of Bread] (Finger) {\textbf{DY-50}};
        \node[draw, fill=gray!20, minimum width=1cm, align=center, below right=0.6cm and 0.3cm of Bread] (Btn) {\textbf{Btn}};
        
        % External 5V
        \node[draw, fill=red!40, minimum width=1cm, align=center, left=2cm of Bread] (Power) {\textbf{5V}};
        
        % Direct connections
        \draw[<->, thick, blue] (Pi) -- node[left] {\tiny direct} (LCD);
        \draw[->, thick, purple] (Cam) -- node[right] {\tiny direct} (Pi);
        
        % Via breadboard
        \draw[<->, thick] (Pi) -- (Bread);
        \draw[->] (Bread) -- (Servos);
        \draw[<->] (Bread) -- (Finger);
        \draw[->] (Bread) -- (Btn);
        \draw[->, red] (Power) -- node[above] {\tiny 5V} (Servos);
        \draw[->, dashed] (Power) |- (Bread);
    \end{tikzpicture}
    \end{center}
    
    \begin{itemize}
        \item \textbf{Connexions directes} : LCD (SPI) + Camera (CSI).
        \item \textbf{Via breadboard soudé} : Servos, Capteur, Bouton + 5V externe.
    \end{itemize}
\end{frame}

% ==============================================================================
% SECTION 4: AGENT ARTIFICIEL (Backend)
% ==============================================================================
\section{Agent Artificiel (Backend)}

\begin{frame}{Core Agent : Architecture Multi-Threadée}
    Le fichier \texttt{main.py} orchestre toutes les opérations matérielles en temps réel.
    
    \vspace{0.3cm}
    \begin{block}{Modèle Producer-Consumer}
        \begin{itemize}
            \item \textbf{Thread Principal} : Gestion UI, GPIO, polling des commandes API.
            \item \textbf{Thread Face Worker} : Scan facial en arrière-plan (non-bloquant).
            \item \textbf{Thread Watchdog} : Surveillance UART (désactivé).
        \end{itemize}
    \end{block}
    
    \vspace{0.2cm}
    \begin{columns}
        \begin{column}{0.48\textwidth}
            \textbf{Gestion de l'Énergie :}
            \begin{itemize}
                \item 30s inactivité $\rightarrow$ Veille auto.
                \item 5 min max $\rightarrow$ Timeout session.
            \end{itemize}
        \end{column}
        \begin{column}{0.48\textwidth}
            \textbf{API FastAPI :}
            \begin{itemize}
                \item \texttt{/users} : CRUD utilisateurs.
                \item \texttt{/command/*} : Unlock, Enroll.
            \end{itemize}
        \end{column}
    \end{columns}
\end{frame}

\begin{frame}[fragile]{Approche Testée : Watchdog (Auto-Guérison)}
    \begin{alertblock}{Problème Identifié}
        Le capteur d'empreintes (UART) peut se figer en cas d'interférences électriques.
    \end{alertblock}
    
    \begin{block}{Solution Envisagée : Thread de Surveillance}
        Un thread vérifie l'état du capteur toutes les 30s et tente un soft-reset.
    \end{block}
    
    \begin{lstlisting}[language=Python, basicstyle=\tiny\ttfamily]
def fingerprint_watchdog():
    while True:
        if finger.read_sysparam() != OK:
            uart.close(); time.sleep(1); uart.open()  # Soft Reset
        time.sleep(30)
    \end{lstlisting}
    
    \begin{exampleblock}{Résultat}
        \textbf{Désactivé} : Le soft-reset ne résout pas les défaillances matérielles réelles.
        $\rightarrow$ Seul un redémarrage physique fonctionne.
    \end{exampleblock}
\end{frame}

\begin{frame}{Base de Données : SQLite Schema}
    Architecture de stockage local avec 4 tables principales :
    
    \vspace{0.2cm}
    \begin{table}[]
        \centering
        \footnotesize
        \rowcolors{2}{gray!10}{white}
        \begin{tabular}{l | l | l}
            \toprule
            \textbf{Table} & \textbf{Colonnes Clés} & \textbf{Rôle} \\
            \midrule
            \texttt{Users} & user\_id, name, face\_encoding, has\_fingerprint & Profils utilisateurs \\
            \texttt{Access\_Logs} & timestamp, event\_type, status & Historique d'accès \\
            \texttt{Pending\_Commands} & command\_type, target\_id, status & File de commandes API \\
            \texttt{System\_Settings} & key\_name, value & Configuration système \\
            \bottomrule
        \end{tabular}
    \end{table}
    
    \vspace{0.2cm}
    \begin{block}{Points Clés}
        \begin{itemize}
            \item \textbf{face\_encoding} : Vecteur 128D stocké en JSON.
            \item \textbf{Pending\_Commands} : Communication asynchrone App $\leftrightarrow$ Hardware.
        \end{itemize}
    \end{block}
\end{frame}

\begin{frame}{Workflow Biométrique}
    \begin{columns}
        \begin{column}{0.48\textwidth}
            \textbf{Enrôlement Visage :}
            \begin{enumerate}
                \item App envoie \texttt{POST /enroll\_face}
                \item Commande insérée en BDD
                \item Core Agent détecte (polling)
                \item Caméra active, détection visage
                \item Encodage 128D $\rightarrow$ SQLite
                \item Confirmation sur écran LCD
            \end{enumerate}
        \end{column}
        \begin{column}{0.48\textwidth}
            \textbf{Enrôlement Empreinte :}
            \begin{enumerate}
                \item App envoie \texttt{POST /enroll\_finger}
                \item Écran LCD : "Placez doigt"
                \item 1ère capture $\rightarrow$ Template 1
                \item "Retirez, placez à nouveau"
                \item 2ème capture $\rightarrow$ Template 2
                \item Fusion modèle $\rightarrow$ Stocké ID
            \end{enumerate}
        \end{column}
    \end{columns}
    
    \vspace{0.3cm}
    \begin{alertblock}{Timeout Sécurité}
        20s (visage) / 30s (empreinte) sans action $\rightarrow$ Annulation automatique.
    \end{alertblock}
\end{frame}

% ==============================================================================
% SECTION 5: ANDROID (Client)
% ==============================================================================
\section{Android (Client)}

\begin{frame}{Application Android : Stack Technique}
    \begin{columns}
        \begin{column}{0.55\textwidth}
            \textbf{Technologies :}
            \begin{itemize}
                \item \textbf{Langage} : Java natif (Android Studio).
                \item \textbf{Réseau} : Retrofit2 + OkHttp.
                \item \textbf{Auth} : Token UUID (SharedPreferences).
                \item \textbf{Min SDK} : API 24 (Android 7.0+).
            \end{itemize}
            
            \vspace{0.3cm}
            \textbf{Fonctionnalités Clés :}
            \begin{itemize}
                \item Auto-Login via Token persistant.
                \item Connexion Wi-Fi one-click (Android 10+).
                \item Enrôlement biométrique à distance.
            \end{itemize}
        \end{column}
        \begin{column}{0.4\textwidth}
            \centering
            \includegraphics[width=0.7\textwidth]{pic/app_dashboard.png}
            \par\vspace{0.2cm}\small Dashboard Utilisateur
        \end{column}
    \end{columns}
\end{frame}

\begin{frame}{Design System : "Vivid Palette"}
    L'application (Java) utilise un code couleur sémantique fort pour guider l'utilisateur :
    
    \begin{columns}
        \begin{column}{0.5\textwidth}
            \begin{itemize}
                \item \textcolor[HTML]{2ECC71}{$\blacksquare$ \textbf{Vert (Emerald)}} : Succès, État "Prêt".
                \item \textcolor[HTML]{F1C40F}{$\blacksquare$ \textbf{Jaune (Sunflower)}} : Action requise, Attente.
                \item \textcolor[HTML]{E57373}{$\blacksquare$ \textbf{Rouge (Soft Red)}} : Danger, Suppression.
            \end{itemize}
            \vspace{0.2cm}
            \textbf{Micro-interactions} :
            \begin{itemize}
                \item Animation "Pop-up" lors de la sélection des canaux.
                \item Masquage dynamique des menus admin.
            \end{itemize}
        \end{column}
        \begin{column}{0.45\textwidth}
            \centering
            \includegraphics[width=0.45\textwidth]{pic/app_dashboard.png} \hfill
            \includegraphics[width=0.45\textwidth]{pic/app_admin.png}
            \par\vspace{0.2cm}\small Dashboard et Admin Panel
        \end{column}
    \end{columns}
\end{frame}

\begin{frame}{Sécurité et Confidentialité}
    \begin{itemize}
        \item \textbf{Protection Admin} : Le système interdit (API + App) la suppression du compte Administrateur (ID 1).
        \item \textbf{Droit à l'oubli} : La suppression du compte nettoie la BDD, le Token et les données biométriques sur le matériel.
    \end{itemize}
\end{frame}

% ==============================================================================
% SECTION 6: INGÉNIERIE (3 min)
% ==============================================================================
\section{Ingénierie et Fabrication}

\begin{frame}{Conception "Code-to-CAD"}
    Le châssis n'est pas dessiné à la souris, mais \textbf{codé} (OpenSCAD / SolidPython).
    
    \begin{columns}
        \begin{column}{0.6\textwidth}
            \begin{itemize}
                \item \textbf{Paramétrique} : Changer \texttt{capsule\_dia = 37} met à jour tout le modèle (base, tiroirs, rails).
                \item \textbf{Modulaire} : Les étages sont imprimés séparément et assemblés.
            \end{itemize}
        \end{column}
        \begin{column}{0.35\textwidth}
            \centering
            \includegraphics[width=\textwidth]{pic/3d_render.png} 
            \par\vspace{0.2cm}\small Rendu OpenSCAD
        \end{column}
    \end{columns}
\end{frame}

% ==============================================================================
% SECTION 7: BILAN (2 min)
% ==============================================================================
\section{Bilan}

\begin{frame}{État Final du Projet}
    \begin{table}[]
        \centering
        \rowcolors{2}{gray!10}{white}
        \begin{tabular}{l | l | l}
            \toprule
            \textbf{Module} & \textbf{Technologie} & \textbf{Statut V1.1} \\
            \midrule
            App Mobile & Android (Java) & \textcolor{green!60!black}{\textbf{Terminé \& Polished}} \\
            Serveur & Python FastAPI & \textcolor{green!60!black}{\textbf{Stable}} \\
            Hardware & Pi 5 / lgpio & \textcolor{green!60!black}{\textbf{Auto-Healing}} \\
            Mécanique & PLA / OpenSCAD & \textcolor{orange}{\textbf{Assemblage}} \\
            \bottomrule
        \end{tabular}
    \end{table}
\end{frame}

\begin{frame}{Démonstration : Captures d'Écran}
    \begin{columns}
        \begin{column}{0.32\textwidth}
            \centering
            \includegraphics[width=0.9\textwidth]{pic/app_login.png}
            \par\vspace{0.2cm}\small \textbf{Login}\\Inscription one-click
        \end{column}
        \begin{column}{0.32\textwidth}
            \centering
            \includegraphics[width=0.9\textwidth]{pic/app_dashboard.png}
            \par\vspace{0.2cm}\small \textbf{Dashboard}\\Accès utilisateur
        \end{column}
        \begin{column}{0.32\textwidth}
            \centering
            \includegraphics[width=0.9\textwidth]{pic/app_admin.png}
            \par\vspace{0.2cm}\small \textbf{Admin}\\Gestion canaux
        \end{column}
    \end{columns}
    
    \vspace{0.3cm}
    \begin{block}{Interface Vivid Palette}
        \begin{itemize}
            \item \textcolor[HTML]{2ECC71}{$\blacksquare$} Vert = Prêt \quad
            \textcolor[HTML]{F1C40F}{$\blacksquare$} Jaune = Sélection \quad
            \textcolor[HTML]{E57373}{$\blacksquare$} Rouge = Danger
        \end{itemize}
    \end{block}
\end{frame}

\begin{frame}{Conclusion}
    DistCapsule a évolué d'un simple concept vers un produit IoT complet.
    \vspace{0.5cm}
    
    \textbf{Points Forts :}
    \begin{itemize}
        \item Expérience utilisateur fluide (App native, Biométrie).
        \item Robustesse technique (Multi-thread, Architecture découplée).
        \item Respect de la vie privée (Données locales).
    \end{itemize}
    
    \vspace{1cm}
    \centering
    \large \textbf{Place à la démonstration !}
\end{frame}

\end{document}
