\documentclass[xcolor=table, aspectratio=169]{beamer}
\usepackage{graphicx}

% --- Core French Language Support ---
\usepackage[T1]{fontenc}
\usepackage[utf8]{inputenc}
\usepackage[french]{babel}
\usepackage{listings}
\usepackage{booktabs} 
\usepackage{tikz} 
\usetikzlibrary{shapes,arrows,positioning}
\usetheme{Darmstadt}
\usecolortheme{dolphin}
\usefonttheme{professionalfonts}

\usepackage{csquotes}

\AtBeginSection[]{
\begin{frame}
    \frametitle{Sommaire}
    \tableofcontents[currentsection]
\end{frame}
}

%% --- Metadata ---
\title[DistCapsule]{DistCapsule : Distributeur Connecté}
\subtitle{Système IoT Biométrique sur Raspberry Pi 5}
\author[Z. Wang \& X. Tang]{Zhaoyu WANG et Xinqi TANG}
\date[\today]{Soutenance Finale -- \today}
\institute[M2 MIASHS]{Université Paris 8 \\ Master Technologies et Handicaps}

%% --- Logo Configuration ---
\titlegraphic{
   \includegraphics[height=1.2cm]{pic/logo-handi.png} \hfill
   \includegraphics[height=1.2cm]{pic/logo-p8.png}
}
\logo{\includegraphics[width=0.8cm]{pic/logo-p8.png}}

%% --- Code Snippet Style ---
\lstset{
    basicstyle=\ttfamily\scriptsize,
    keywordstyle=\color{blue!80!black}\bfseries,
    stringstyle=\color{green!60!black},
    commentstyle=\color{gray}\itshape,
    numbers=left,
    numberstyle=\tiny\color{gray},
    frame=shadowbox,
    rulesepcolor=\color{gray},
    breaklines=true,
    language=Python,
    showstringspaces=false
}

\begin{document}

{
\logo{}
\begin{frame}
    \titlepage   
\end{frame}
}

% --- Global Table of Contents ---
\begin{frame}{Plan de la présentation}
    \tableofcontents
\end{frame}

% ==============================================================================
% SECTION 1: CONTEXTE (3-4 min)
% ==============================================================================
\section{Contexte et Problématique}

\begin{frame}{Le Constat}
    Dans un espace café partagé, les capsules sont souvent mélangées en vrac ou stockées individuellement, encombrant inutilement le plan de travail.
    
    \begin{columns}
        \begin{column}{0.6\textwidth}
            \begin{alertblock}{Problèmes Identifiés}
                \begin{itemize}
                    \item \textbf{Désordre} : Capsules en vrac, boîtes empilées, perte d'espace sur le plan de travail.
                    \item \textbf{Confusion} : Difficulté à distinguer ses propres capsules (variétés similaires).
                    \item \textbf{Stockage} : Certains utilisateurs cachent leurs capsules pour éviter le vol ou le mélange.
                \end{itemize}
            \end{alertblock}
        \end{column}
        \begin{column}{0.38\textwidth}
            \begin{figure}
                \includegraphics[width=\textwidth,height=4cm,keepaspectratio]{pic/amazon_ref.jpg}
                \caption{Situation actuelle (Stockage passif)}
            \end{figure}
        \end{column}
    \end{columns}
\end{frame}

\begin{frame}{Notre Solution : DistCapsule}
    Transformer un simple présentoir en un \textbf{écosystème IoT} intelligent.
    
    \vspace{0.5cm}
    \begin{block}{Proposition de Valeur}
        \begin{itemize}
            \item \textbf{Centralisation} : Une tour compacte stockant 5 variétés.
            \item \textbf{Personnalisation} : Chaque utilisateur a ses droits d'accès.
            \item \textbf{Fluidité} : Distribution sans contact (Biométrie) ou via Mobile.
        \end{itemize}
    \end{block}
\end{frame}

% ==============================================================================
% SECTION 2: ARCHITECTURE (5 min)
% ==============================================================================
\section{Architecture Technique}

\begin{frame}{Architecture 3-Tiers}
    Pour garantir la stabilité, nous avons découplé les responsabilités en trois couches distinctes :
    
    \vspace{0.3cm}
    \begin{center}
    \begin{tikzpicture}[node distance=2.5cm, auto, thick]
        % Nodes
        \node[draw, fill=green!20, minimum size=1cm, align=center] (App) {\textbf{Android App}\\(Java)};
        \node[draw, fill=blue!20, minimum size=1cm, align=center, right of=App, node distance=5cm] (API) {\textbf{API Server}\\(FastAPI)};
        \node[draw, fill=red!20, minimum size=1cm, align=center, right of=API, node distance=5cm] (Core) {\textbf{Core Agent}\\(Python)};
        
        % Arrows
        \draw[<->] (App) -- node[above] {HTTP/JSON} node[below] {Token} (API);
        \draw[<->] (API) -- node[above] {SQLite} node[below] {Queue} (Core);
    \end{tikzpicture}
    \end{center}
    
    \begin{itemize}
        \item \textbf{App} : Interface de commande (Client Léger).
        \item \textbf{API} : Passerelle de sécurité et gestion des données.
        \item \textbf{Core} : Gestion du matériel (Boucle infinie, non-bloquante).
    \end{itemize}
\end{frame}

\begin{frame}{Flux de Données (Workflow)}
    Exemple : Un utilisateur demande une capsule via l'application.
    \vspace{0.2cm}
    \begin{columns}[T]
    \begin{column}{0.35\textwidth}    
    \begin{enumerate}
        \item \textbf{Requête} : \par\texttt{POST} avec Token.
        \item \textbf{Validation} : \par Auth \& Insertion BDD.
        \item \textbf{Exécution} : \par \textbf{Core Agent} (Polling).
        \item \textbf{Action} : \par Pilotage Servo (\texttt{lgpio}).
        \item \textbf{Feedback} : \par Feedback IPS \& App.
    \end{enumerate}
    \end{column}
    \begin{column}{0.65\textwidth}
    \begin{figure}
        \centering
        \includegraphics[width=0.9\textwidth,height=5cm,keepaspectratio]{pic/data_flow.png}
        \par\vspace{0.2cm}\small Schéma du flux de données entre les composants
    \end{figure}

    \end{column}
    \end{columns}
\end{frame}

\begin{frame}{Réseau : Stabilité et Connectivité}
    \begin{block}{Le Défi}
        Les smartphones modernes refusent souvent les connexions Wi-Fi "sans internet" (No Gateway).
    \end{block}
    
    \begin{exampleblock}{Solution Finale (V1.1)}
        \begin{itemize}
            \item \textbf{Mode} : Hotspot Standard avec passerelle DHCP activée.
            \item \textbf{Avantage} : Connexion immédiate et stable sur 100% des appareils Android/iOS testés.
            \item \textbf{UX} : Bouton "Connecter au Wi-Fi" intégré directement dans l'application.
        \end{itemize}
    \end{exampleblock}
\end{frame}

% ==============================================================================
% SECTION 3: DÉFIS TECHNIQUES (8 min)
% ==============================================================================
\section{Défis Techniques et Solutions}

\begin{frame}{1. Hardware : La Révolution Pi 5}
    Le passage au Raspberry Pi 5 a cassé la compatibilité de nombreuses bibliothèques historiques.
    
    \begin{itemize}
        \item \textbf{Obsolescence} : \texttt{RPi.GPIO} ne fonctionne pas sur la puce RP1 du Pi 5.
        \item \textbf{Solution} : Migration vers \textbf{\texttt{lgpio}} (Linux GPIO).
    \end{itemize}
    
    \vspace{0.3cm}
    \begin{block}{Innovation : Soft-PWM}
        Les bibliothèques standard de servos n'étant pas prêtes, nous avons écrit notre propre contrôleur PWM logiciel pour piloter les 5 moteurs SG90 avec précision, sans jitter.
    \end{block}
\end{frame}

\begin{frame}[fragile]{2. Fiabilité : Le "Watchdog" (Auto-Guérison)}
    Le capteur d'empreintes (UART) peut parfois se figer (timeout) en cas d'interférences électriques.
    
    \begin{block}{Implémentation d'un Thread de Surveillance}
        Un thread indépendant vérifie l'état du capteur toutes les 30 secondes.
    \end{block}
    
    \begin{lstlisting}[language=Python, basicstyle=\tiny\ttfamily]
def fingerprint_watchdog():
    while True:
        try:
            if finger.read_sysparam() != OK: raise Error
        except:
            # Soft Reset : Fermer/Rouvrir le port Serie
            uart.close(); time.sleep(1); uart.open()
        time.sleep(30)
    \end{lstlisting}
\end{frame}

% ==============================================================================
% SECTION 4: EXPÉRIENCE MOBILE (5 min)
% ==============================================================================
\section{Expérience Mobile (Android)}

\begin{frame}{Design System : "Vivid Palette"}
    L'application (Java) utilise un code couleur sémantique fort pour guider l'utilisateur :
    
    \begin{columns}
        \begin{column}{0.5\textwidth}
            \begin{itemize}
                \item \textcolor[HTML]{2ECC71}{$\blacksquare$ \textbf{Vert (Emerald)}} : Succès, État "Prêt".
                \item \textcolor[HTML]{F1C40F}{$\blacksquare$ \textbf{Jaune (Sunflower)}} : Action requise, Attente.
                \item \textcolor[HTML]{E57373}{$\blacksquare$ \textbf{Rouge (Soft Red)}} : Danger, Suppression.
            \end{itemize}
            \vspace{0.2cm}
            \textbf{Micro-interactions} :
            \begin{itemize}
                \item Animation "Pop-up" lors de la sélection des canaux.
                \item Masquage dynamique des menus admin.
            \end{itemize}
        \end{column}
        \begin{column}{0.45\textwidth}
            \centering
            \includegraphics[width=0.45\textwidth]{pic/app_dashboard.png} \hfill
            \includegraphics[width=0.45\textwidth]{pic/app_admin.png}
            \par\vspace{0.2cm}\small Dashboard et Admin Panel
        \end{column}
    \end{columns}
\end{frame}

\begin{frame}{Sécurité et Confidentialité}
    \begin{itemize}
        \item \textbf{Protection Admin} : Le système interdit (API + App) la suppression du compte Administrateur (ID 1).
        \item \textbf{Droit à l'oubli} : La suppression du compte nettoie la BDD, le Token et les données biométriques sur le matériel.
    \end{itemize}
\end{frame}

% ==============================================================================
% SECTION 5: INGÉNIERIE (3 min)
% ==============================================================================
\section{Ingénierie et Fabrication}

\begin{frame}{Conception "Code-to-CAD"}
    Le châssis n'est pas dessiné à la souris, mais \textbf{codé} (OpenSCAD / SolidPython).
    
    \begin{columns}
        \begin{column}{0.6\textwidth}
            \begin{itemize}
                \item \textbf{Paramétrique} : Changer \texttt{capsule\_dia = 37} met à jour tout le modèle (base, tiroirs, rails).
                \item \textbf{Modulaire} : Les étages sont imprimés séparément et assemblés.
            \end{itemize}
        \end{column}
        \begin{column}{0.35\textwidth}
            \centering
            \includegraphics[width=\textwidth]{pic/3d_render.png} 
            \par\vspace{0.2cm}\small Rendu OpenSCAD
        \end{column}
    \end{columns}
\end{frame}

\begin{frame}{Câblage et Intégration}
    \begin{figure}
        \centering
        \includegraphics[width=0.8\textwidth,height=4cm,keepaspectratio]{pic/hardware_wiring.png}
        \par\vspace{0.2cm}\small Intégration dense : Pi 5 + 5 Servos + Écran + Caméra
    \end{figure}
    
    \begin{itemize}
        \item Alimentation séparée pour les moteurs (stabilité du Pi).
        \item Gestion des conflits de pins GPIO (PWM vs SPI vs UART).
    \end{itemize}
\end{frame}

% ==============================================================================
% SECTION 6: BILAN (2 min)
% ==============================================================================
\section{Bilan}

\begin{frame}{État Final du Projet}
    \begin{table}[]
        \centering
        \rowcolors{2}{gray!10}{white}
        \begin{tabular}{l | l | l}
            \toprule
            \textbf{Module} & \textbf{Technologie} & \textbf{Statut V1.1} \\
            \midrule
            App Mobile & Android (Java) & \textcolor{green!60!black}{\textbf{Terminé \& Polished}} \\
            Serveur & Python FastAPI & \textcolor{green!60!black}{\textbf{Stable}} \\
            Hardware & Pi 5 / lgpio & \textcolor{green!60!black}{\textbf{Auto-Healing}} \\
            Mécanique & PLA / OpenSCAD & \textcolor{orange}{\textbf{Assemblage}} \\
            \bottomrule
        \end{tabular}
    \end{table}
\end{frame}

\begin{frame}{Conclusion}
    DistCapsule a évolué d'un simple concept vers un produit IoT complet.
    \vspace{0.5cm}
    
    \textbf{Points Forts :}
    \begin{itemize}
        \item Expérience utilisateur fluide (App native, Biométrie).
        \item Robustesse technique (Watchdog, Architecture découplée).
        \item Respect de la vie privée (Données locales).
    \end{itemize}
    
    \vspace{1cm}
    \centering
    \large \textbf{Place à la démonstration !}
\end{frame}

\end{document}
