\documentclass[xcolor=table, aspectratio=169]{beamer}
\usepackage{graphicx}

% --- Core French Language Support ---
\usepackage[T1]{fontenc}
\usepackage[utf8]{inputenc}
\usepackage[french]{babel}
\usepackage{listings}
\usepackage{booktabs} 
\usepackage{tikz} 
\usetikzlibrary{shapes,arrows,positioning}
\usetheme{Darmstadt}
\usecolortheme{dolphin}
\usefonttheme{professionalfonts}

\usepackage{csquotes}

\AtBeginSection[]{
\begin{frame}
    \frametitle{Sommaire}
    \tableofcontents[currentsection]
\end{frame}
}

%% --- Metadata ---
\title[DistCapsule]{DistCapsule : Distributeur Connecté}
\subtitle{Système Embarqué Connecté sur Raspberry Pi 5}
\author[Z. Wang \& X. Tang]{Zhaoyu WANG et Xinqi TANG}
\date[\today]{Soutenance Finale -- \today}
\institute[M2 MIASHS]{Université Paris 8 \\ Master Technologies et Handicaps}

%% --- Logo Configuration ---
\titlegraphic{
   \includegraphics[height=1.2cm]{pic/logo-handi.png} \hfill
   \includegraphics[height=1.2cm]{pic/logo-p8.png}
}
\logo{\includegraphics[width=0.8cm]{pic/logo-p8.png}}

%% --- Code Snippet Style ---
\lstset{
    basicstyle=\ttfamily\scriptsize,
    keywordstyle=\color{blue!80!black}\bfseries,
    stringstyle=\color{green!60!black},
    commentstyle=\color{gray}\itshape,
    numbers=left,
    numberstyle=\tiny\color{gray},
    frame=shadowbox,
    rulesepcolor=\color{gray},
    breaklines=true,
    language=Python,
    showstringspaces=false
}

\begin{document}

{
\logo{}
\begin{frame}
    \titlepage   
\end{frame}
}

% --- Global Table of Contents ---
\begin{frame}{Plan de la présentation}
    \tableofcontents
\end{frame}

% ==============================================================================
% SECTION 1: CONTEXTE (3-4 min)
% ==============================================================================
\section{Contexte et Problématique}

\begin{frame}{Le Constat}
    Dans un espace café partagé, les capsules sont souvent mélangées en vrac ou stockées individuellement, encombrant inutilement le plan de travail.
    
    \begin{columns}
        \begin{column}{0.6\textwidth}
            \begin{alertblock}{Problèmes Identifiés}
                \begin{itemize}
                    \item \textbf{Désordre} : Capsules en vrac, boîtes empilées, perte d'espace sur le plan de travail.
                    \item \textbf{Confusion} : Difficulté à distinguer ses propres capsules (variétés similaires).
                    \item \textbf{Stockage} : Certains utilisateurs cachent leurs capsules pour éviter le vol ou le mélange.
                \end{itemize}
            \end{alertblock}
        \end{column}
        \begin{column}{0.38\textwidth}
            \begin{figure}
                \includegraphics[width=\textwidth,height=4cm,keepaspectratio]{pic/espace_cafe.png}
                \caption{Situation actuelle (Stockage passif)}
            \end{figure}
        \end{column}
    \end{columns}
\end{frame}

\begin{frame}{Notre Solution : DistCapsule}
    Transformer un simple présentoir en un \textbf{système embarqué} intelligent.
    
    \vspace{0.3cm}
    \begin{columns}
        \begin{column}{0.6\textwidth}
            \begin{block}{Proposition de Valeur}
                \begin{itemize}
                    \item \textbf{Centralisation} : Une tour compacte stockant 5 variétés.
                    \item \textbf{Personnalisation} : Chaque utilisateur a ses droits d'accès.
                    \item \textbf{Fluidité} : Distribution Biométrique ou via Mobile.
                \end{itemize}
            \end{block}
        \end{column}
        \begin{column}{0.35\textwidth}
            \centering
            \includegraphics[width=\textwidth,height=4cm,keepaspectratio]{pic/amazon_ref.jpg}
            \par\vspace{0.2cm}\small Présentoir à transformer
        \end{column}
    \end{columns}
\end{frame}

% ==============================================================================
% SECTION 2: ARCHITECTURE (5 min)
% ==============================================================================
\section{Architecture Technique}

\begin{frame}{Architecture AAA (3-Tiers)}
    Pour garantir la stabilité, nous avons découplé les responsabilités en trois couches distinctes :
    
    \vspace{0.3cm}
    \begin{center}
    \begin{tikzpicture}[node distance=2.5cm, auto, thick]
        % Nodes - AAA Convention
        \node[draw, fill=green!20, minimum size=1cm, align=center] (Android) {\textbf{Android}\\(Client)};
        \node[draw, fill=blue!20, minimum size=1cm, align=center, right of=Android, node distance=5cm] (Agent) {\textbf{Agent Artificiel}\\(FastAPI)};
        \node[draw, fill=red!20, minimum size=1cm, align=center, right of=Agent, node distance=5cm] (Arduino) {\textbf{Arduino}\\(Raspberry Pi 5)};
        
        % Arrows - Simplified labels
        \draw[<->] (Android) -- node[above] {REST API} (Agent);
        \draw[<->] (Agent) -- node[above] {Commands} (Arduino);
    \end{tikzpicture}
    \end{center}
    
    \begin{itemize}
        \item \textbf{Android} : Interface utilisateur mobile (Java).
        \item \textbf{Agent} : Passerelle de sécurité et orchestration (Python).
        \item \textbf{Arduino} : Contrôle matériel temps réel (lgpio, Servos, Capteurs).
    \end{itemize}
\end{frame}

\begin{frame}{Flux de Données (Workflow)}
    Exemple : Un utilisateur demande une capsule via l'application.
    \vspace{0.2cm}
    \begin{columns}[T]
    \begin{column}{0.35\textwidth}    
    \begin{enumerate}
        \item \textbf{Requête} : \par\texttt{POST} avec Token.
        \item \textbf{Validation} : \par Auth \& Insertion BDD.
        \item \textbf{Exécution} : \par \textbf{Core Agent} (Polling).
        \item \textbf{Action} : \par Pilotage Servo (\texttt{lgpio}).
        \item \textbf{Feedback} : \par Feedback IPS \& App.
    \end{enumerate}
    \end{column}
    \begin{column}{0.65\textwidth}
    \begin{figure}
        \centering
        \includegraphics[width=0.9\textwidth,height=5cm,keepaspectratio]{pic/data_flow.png}
        \par\vspace{0.2cm}\small Schéma du flux de données entre les composants
    \end{figure}

    \end{column}
    \end{columns}
\end{frame}

\begin{frame}{Réseau : Topologie et Configuration}
    \begin{center}
    \begin{tikzpicture}[node distance=2.5cm, auto, thick]
        % Nodes
        \node[draw, fill=green!20, minimum size=0.8cm, align=center] (Phone) {\textbf{Android}\\App};
        \node[draw, fill=red!20, minimum width=2.5cm, minimum height=1.2cm, align=center, right of=Phone, node distance=5cm] (Pi) {\textbf{Raspberry Pi 5}\\wlan0 (Hotspot) + eth0};
        \node[draw, fill=blue!20, minimum size=0.8cm, align=center, right of=Pi, node distance=5cm] (PC) {\textbf{PC Admin}\\Internet + SSH};
        
        % Arrows
        \draw[<->] (Phone) -- node[above] {\scriptsize Wi-Fi} node[below] {\scriptsize 192.168.4.x} (Pi);
        \draw[<->] (Pi) -- node[above] {\scriptsize RJ45} node[below] {\scriptsize 192.168.3.x} (PC);
    \end{tikzpicture}
    \end{center}
    
    \vspace{0.2cm}
    \begin{columns}
        \begin{column}{0.48\textwidth}
            \textbf{Interface Hotspot (wlan0) :}
            \begin{itemize}
                \item \textbf{SSID} : \texttt{DistCapsule\_Box}
                \item \textbf{IP} : \texttt{192.168.4.1}
                \item \textbf{Port API} : \texttt{8000}
                \item \textbf{Sécurité} : WPA2 + Token
            \end{itemize}
        \end{column}
        \begin{column}{0.48\textwidth}
            \textbf{Interface Ethernet (eth0) :}
            \begin{itemize}
                \item \textbf{IP} : \texttt{192.168.3.14}
                \item \textbf{Usage} : SSH, Internet (Git, Pip install, Mises à jour)
                \item \textbf{Internet} : Accès via PC (ICS)
            \end{itemize}
        \end{column}
    \end{columns}
\end{frame}

% ==============================================================================
% SECTION 3: ARDUINO (Matériel)
% ==============================================================================
\section{Arduino (Raspberry Pi 5 - Matériel)}

\begin{frame}{Raspberry Pi 5 : Composants Intégrés}
    \begin{columns}
        \begin{column}{0.57\textwidth}
            \textbf{Périphériques Connectés :}
            \begin{itemize}
                \item \textbf{5x Servos SG90} : PWM logiciel, GPIO 18/6/12/13/19.
                \item \textbf{DY-50} : Empreintes digitales (UART 57600).
                \item \textbf{Camera Module 3} : Reconnaissance faciale (CSI, IMX708).
                \item \textbf{ST7789 LCD} : Écran 240×240 (SPI).
                \item \textbf{Bouton Wake-Up} : Réveil système (GPIO 26).
            \end{itemize}
            
            \begin{block}{Alimentation}
                Pi 5 : USB-C 27W \par Servos : 5V externe (masse commune)
            \end{block}
        \end{column}
        \begin{column}{0.4\textwidth}
            \centering
            \includegraphics[width=\textwidth,height=4cm,keepaspectratio]{pic/hardware_wiring.jpg}
            \par\vspace{0.2cm}\small Câblage réel du prototype
        \end{column}
    \end{columns}
\end{frame}

\begin{frame}{Défi Hardware : La Révolution Pi 5}
    Le passage au Raspberry Pi 5 a cassé la compatibilité de nombreuses bibliothèques historiques.
    
    \begin{itemize}
        \item \textbf{Obsolescence} : \texttt{RPi.GPIO} ne fonctionne pas sur la puce RP1 du Pi 5.
        \item \textbf{Solution} : Migration vers \textbf{\texttt{lgpio}} (Linux GPIO).
    \end{itemize}
    
    \vspace{0.3cm}
    \begin{block}{Innovation : Soft-PWM}
        Pi 5 : 4 canaux HW-PWM (1 réservé ventilateur) $\rightarrow$ Soft-PWM pour 5 servos.
    \end{block}
\end{frame}

\begin{frame}{Câblage et Intégration}
    \begin{center}
    \begin{tikzpicture}[node distance=1cm, auto, thick, scale=0.7, every node/.style={scale=0.7}]
        % Central Pi
        \node[draw, fill=red!20, minimum width=2cm, minimum height=1cm, align=center] (Pi) {\textbf{Pi 5}\\GPIO};
        
        % Direct connections (no breadboard)
        \node[draw, fill=blue!20, minimum width=1.2cm, align=center, above left=0.8cm and 1.2cm of Pi] (LCD) {\textbf{LCD}\\SPI};
        \node[draw, fill=purple!20, minimum width=1.2cm, align=center, above right=0.8cm and 1.2cm of Pi] (Cam) {\textbf{Cam}\\CSI};
        
        % Breadboard (soldered)
        \node[draw, fill=yellow!40, minimum width=3cm, minimum height=0.6cm, align=center, below=1.2cm of Pi] (Bread) {\textbf{Breadboard} (soudé)};
        
        % Through breadboard
        \node[draw, fill=orange!20, minimum width=1.5cm, align=center, below left=0.6cm and 0.3cm of Bread] (Servos) {\textbf{5x SG90}};
        \node[draw, fill=green!20, minimum width=1.2cm, align=center, below=0.6cm of Bread] (Finger) {\textbf{DY-50}};
        \node[draw, fill=gray!20, minimum width=1cm, align=center, below right=0.6cm and 0.3cm of Bread] (Btn) {\textbf{Btn}};
        
        % External 5V
        \node[draw, fill=red!40, minimum width=1cm, align=center, left=2cm of Bread] (Power) {\textbf{5V}};
        
        % Direct connections
        \draw[<->, thick, blue] (Pi) -- node[left] {\tiny direct} (LCD);
        \draw[->, thick, purple] (Cam) -- node[right] {\tiny direct} (Pi);
        
        % Via breadboard
        \draw[<->, thick] (Pi) -- (Bread);
        \draw[->] (Bread) -- (Servos);
        \draw[<->] (Bread) -- (Finger);
        \draw[->] (Bread) -- (Btn);
        \draw[->, red] (Power) -- node[above] {\tiny 5V} (Servos);
        \draw[->, dashed] (Power) |- (Bread);
    \end{tikzpicture}
    \end{center}
    
    \begin{itemize}
        \item \textbf{Connexions directes} : LCD (SPI) + Camera (CSI).
        \item \textbf{Via breadboard} : Servos, Capteur, Bouton + 5V externe + \textbf{condo 500\textmu F}.
    \end{itemize}
\end{frame}

% ==============================================================================
% SECTION 4: AGENT ARTIFICIEL (Backend)
% ==============================================================================
\section{Agent Artificiel (Backend)}

\begin{frame}{Core Agent : Architecture Multi-Threadée}
    Le fichier \texttt{main.py} orchestre toutes les opérations matérielles en temps réel.
    
    \vspace{0.3cm}
    \begin{block}{Modèle Producer-Consumer}
        \begin{itemize}
            \item \textbf{Thread Principal} : Gestion UI, GPIO, polling des commandes API.
            \item \textbf{Thread Face Worker} : Scan facial en arrière-plan (non-bloquant).
            \item \textbf{Thread Watchdog} : Surveillance UART (désactivé).
        \end{itemize}
    \end{block}
    
    \vspace{0.2cm}
    \begin{columns}
        \begin{column}{0.48\textwidth}
            \textbf{Gestion de l'Énergie :}
            \begin{itemize}
                \item 30s inactivité $\rightarrow$ Veille auto.
                \item 5 min max $\rightarrow$ Timeout session.
            \end{itemize}
        \end{column}
        \begin{column}{0.48\textwidth}
            \textbf{API FastAPI :}
            \begin{itemize}
                \item \texttt{/users} : CRUD utilisateurs.
                \item \texttt{/command/*} : Unlock, Enroll.
            \end{itemize}
        \end{column}
    \end{columns}
\end{frame}

\begin{frame}{Approche Testée : Watchdog (Auto-Guérison)}
    \begin{alertblock}{Problème Identifié}
        Le capteur d'empreintes (UART) peut se bloquer de façon aléatoire (cause inconnue).
    \end{alertblock}
    
    \begin{block}{Solution Envisagée}
        Thread de surveillance : vérifie l'état toutes les 30s, tente un soft-reset UART.
    \end{block}
    
    \begin{exampleblock}{Résultat \& Contournement}
        \textbf{Désactivé} : Le soft-reset ne résout pas le blocage. \\
        $\rightarrow$ \textbf{Fallback} : Reconnaissance faciale + App restent fonctionnels. \\
        $\rightarrow$ Seul un \textbf{redémarrage électrique} restaure le capteur.
    \end{exampleblock}
\end{frame}

\begin{frame}{Base de Données : SQLite Schema}
    Architecture de stockage local avec 4 tables principales :
    
    \vspace{0.3cm}
    \begin{table}[]
        \centering
        \footnotesize
        \rowcolors{2}{gray!10}{white}
        \begin{tabular}{l | l | l}
            \toprule
            \textbf{Table} & \textbf{Rôle} & \textcolor{gray}{\textit{Notes}} \\
            \midrule
            \texttt{Users} & Profils utilisateurs & \textcolor{gray}{\scriptsize face\_encoding, has\_fingerprint} \\
            \texttt{Access\_Logs} & Historique d'accès & \textcolor{gray}{\scriptsize timestamp, event\_type} \\
            \texttt{Pending\_Commands} & File commandes API & \textcolor{gray}{\scriptsize status: pending/completed} \\
            \texttt{System\_Settings} & Configuration & \textcolor{gray}{\scriptsize clé-valeur} \\
            \bottomrule
        \end{tabular}
    \end{table}
    
    \vspace{0.3cm}
    \begin{block}{Communication App $\leftrightarrow$ Hardware}
        Via \texttt{Pending\_Commands} : App insère $\rightarrow$ Core Agent poll $\rightarrow$ Exécute.
    \end{block}
\end{frame}

\begin{frame}{Workflow Biométrique}
    \begin{center}
    \begin{tikzpicture}[node distance=0.6cm, auto, thick, scale=0.75, every node/.style={scale=0.75}]
        % Face enrollment flow (left)
        \node[draw, fill=green!20, minimum width=2cm, align=center] (F1) {\textbf{Visage}};
        \node[draw, fill=blue!10, below=of F1, minimum width=2cm, align=center] (F2) {App: POST};
        \node[draw, fill=blue!10, below=of F2, minimum width=2cm, align=center] (F3) {Caméra ON};
        \node[draw, fill=blue!10, below=of F3, minimum width=2cm, align=center] (F4) {Détection};
        \node[draw, fill=blue!10, below=of F4, minimum width=2cm, align=center] (F5) {Encode 128D};
        \node[draw, fill=green!30, below=of F5, minimum width=2cm, align=center] (F6) {SQLite};
        
        \draw[->] (F1) -- (F2);
        \draw[->] (F2) -- (F3);
        \draw[->] (F3) -- (F4);
        \draw[->] (F4) -- (F5);
        \draw[->] (F5) -- (F6);
        
        % Fingerprint enrollment flow (right)
        \node[draw, fill=orange!20, minimum width=2cm, align=center, right=3cm of F1] (P1) {\textbf{Empreinte}};
        \node[draw, fill=blue!10, below=of P1, minimum width=2cm, align=center] (P2) {App: POST};
        \node[draw, fill=blue!10, below=of P2, minimum width=2cm, align=center] (P3) {1ère capture};
        \node[draw, fill=blue!10, below=of P3, minimum width=2cm, align=center] (P4) {Retirer doigt};
        \node[draw, fill=blue!10, below=of P4, minimum width=2cm, align=center] (P5) {2ème capture};
        \node[draw, fill=orange!30, below=of P5, minimum width=2cm, align=center] (P6) {Stocké ID};
        
        \draw[->] (P1) -- (P2);
        \draw[->] (P2) -- (P3);
        \draw[->] (P3) -- (P4);
        \draw[->] (P4) -- (P5);
        \draw[->] (P5) -- (P6);
        
        % Timeout note
        \node[draw, fill=red!20, minimum width=5cm, align=center, below=0.8cm of F6, xshift=2.5cm] (T) {\textbf{Timeout} : 20s (visage) / 30s (empreinte)};
    \end{tikzpicture}
    \end{center}
\end{frame}

% ==============================================================================
% SECTION 5: ANDROID (Client)
% ==============================================================================
\section{Android (Client)}

\begin{frame}{Application Android : Stack Technique}
    \begin{columns}
        \begin{column}{0.55\textwidth}
            \textbf{Technologies :}
            \begin{itemize}
                \item \textbf{Langage} : Java natif (Android Studio).
                \item \textbf{Réseau} : Retrofit2 + OkHttp.
                \item \textbf{Auth} : Token UUID (SharedPreferences).
                \item \textbf{Min SDK} : API 24 (Android 7.0+).
            \end{itemize}
            
            \vspace{0.3cm}
            \textbf{Fonctionnalités Clés :}
            \begin{itemize}
                \item Auto-Login via Token persistant.
                \item Connexion Wi-Fi one-click (Android 10+).
                \item Enrôlement biométrique à distance.
            \end{itemize}
        \end{column}
        \begin{column}{0.4\textwidth}
            \centering
            \includegraphics[width=0.5\textwidth]{pic/app_dashboard.png}
            \par\vspace{0.1cm}\small Dashboard Utilisateur
        \end{column}
    \end{columns}
\end{frame}

\begin{frame}{Design System : "Vivid Palette"}
    L'application (Java) utilise un code couleur sémantique fort pour guider l'utilisateur :
    
    \begin{columns}
        \begin{column}{0.5\textwidth}
            \begin{itemize}
                \item \textcolor[HTML]{2ECC71}{$\blacksquare$ \textbf{Vert (Emerald)}} : Succès, État "Prêt".
                \item \textcolor[HTML]{F1C40F}{$\blacksquare$ \textbf{Jaune (Sunflower)}} : Action requise, Attente.
                \item \textcolor[HTML]{E57373}{$\blacksquare$ \textbf{Rouge (Soft Red)}} : Danger, Suppression.
            \end{itemize}
            \vspace{0.2cm}
            \textbf{Micro-interactions} :
            \begin{itemize}
                \item Animation "Pop-up" lors de la sélection des canaux.
                \item Masquage dynamique des menus admin.
            \end{itemize}
        \end{column}
        \begin{column}{0.45\textwidth}
            \centering
            \includegraphics[width=0.45\textwidth]{pic/app_dashboard.png} \hfill
            \includegraphics[width=0.45\textwidth]{pic/app_admin.png}
            \par\vspace{0.2cm}\small Dashboard et Admin Panel
        \end{column}
    \end{columns}
\end{frame}

\begin{frame}{UX : Parcours Utilisateur}
    \begin{columns}
        \begin{column}{0.48\textwidth}
            \textbf{Nouvel Utilisateur :}
            \begin{enumerate}
                \item Lancement App
                \item Token absent $\rightarrow$ Inscription
                \item Nom + Génération Token UUID
                \item Redirection vers Dashboard
            \end{enumerate}
            
            \vspace{0.3cm}
            \textbf{Utilisateur Existant :}
            \begin{enumerate}
                \item Lancement App
                \item Token détecté $\rightarrow$ Auto-Login
                \item Accès direct Dashboard
            \end{enumerate}
        \end{column}
        \begin{column}{0.48\textwidth}
            \textbf{Interface selon Rôle :}
            
            \vspace{0.2cm}
            \textbf{Admin (auth\_level $\leq$ 1) :}
            \begin{itemize}
                \item Gestion des canaux
                \item Création/Suppression users
                \item Enrôlement biométrique
            \end{itemize}
            
            \vspace{0.2cm}
            \textbf{Utilisateur Standard :}
            \begin{itemize}
                \item Déverrouillage One-Tap
                \item Gestion de son profil
            \end{itemize}
        \end{column}
    \end{columns}
\end{frame}

\begin{frame}{Sécurité et Confidentialité}
    \begin{columns}
        \begin{column}{0.48\textwidth}
            \textbf{Protection des Données :}
            \begin{itemize}
                \item \textbf{Stockage local} : Aucun cloud, tout sur Pi.
                \item \textbf{Biométrie} : Encodage 128D (non-réversible).
                \item \textbf{Token UUID} : Auth sans mot de passe.
            \end{itemize}
        \end{column}
        \begin{column}{0.48\textwidth}
            \textbf{Règles Métier :}
            \begin{itemize}
                \item \textbf{Admin protégé} : auth\_level $\leq$ 1 non-supprimable.
                \item \textbf{Droit à l'oubli} : Suppression complète (BDD + capteur).
                \item \textbf{Isolation} : Hotspot local, pas d'internet.
            \end{itemize}
        \end{column}
    \end{columns}
    
    \vspace{0.3cm}
    \begin{block}{RGPD : Conformité par Design}
        Données biométriques jamais transmises $\rightarrow$ Traitement 100\% local.
    \end{block}
\end{frame}

% ==============================================================================
% SECTION 6: INGÉNIERIE (3 min)
% ==============================================================================
\section{Ingénierie et Fabrication}

\begin{frame}{Conception 3D}
    \begin{columns}
        \begin{column}{0.5\textwidth}
            \textbf{Modélisation 3D :}
            \begin{itemize}
                \item Code Python (SolidPython2)
                \item Génère fichiers \texttt{.scad}
                \item Rendu via OpenSCAD
            \end{itemize}
            
            \vspace{0.3cm}
            \textbf{Fabrication :}
            \begin{itemize}
                \item \textbf{Matériau} : PLA
                \item \textbf{Temps} : $\sim$4 jours d'impression
                \item \textbf{Pièces} : 8 modules assemblés
                \item \textbf{Dimensions} : 260 × 260 × 15 mm
            \end{itemize}
        \end{column}
        \begin{column}{0.45\textwidth}
            \centering
            \includegraphics[width=\textwidth]{pic/3d_render.png} 
            \par\vspace{0.2cm}\small Rendu OpenSCAD
        \end{column}
    \end{columns}
\end{frame}

\begin{frame}{Impression 3D : Avancement}
    \begin{columns}
        \begin{column}{0.55\textwidth}
            \textbf{État des Pièces :}
            \begin{table}[]
                \footnotesize
                \rowcolors{2}{gray!10}{white}
                \begin{tabular}{l | l}
                    \toprule
                    \textbf{Pièce} & \textbf{Statut} \\
                    \midrule
                    Base & \textcolor{green!60!black}{\textbf{OK}} \\
                    Tiroir & \textcolor{green!60!black}{\textbf{OK}} \\
                    Bras servo (×5) & \textcolor{orange}{\textbf{2/5}} \\
                    Poussoir (test) & \textcolor{green!60!black}{\textbf{OK}} \\
                    Support périph. & \textcolor{gray}{À concevoir} \\
                    \bottomrule
                \end{tabular}
            \end{table}
        \end{column}
        \begin{column}{0.4\textwidth}
            \textbf{Inspiration :}
            \par Amazon "Capsule Rangement"
            
            \vspace{0.5cm}
            \textbf{Poussoir :}
            \par Pièce expérimentale pour éjecter les capsules automatiquement.
        \end{column}
    \end{columns}
    
    \vspace{0.2cm}
    \begin{block}{Leçon Apprise}
        Imprimer de \textbf{petits prototypes} d'abord pour tester dimensions/assemblage. \\
        \textit{Conseil précieux de Dom — gain de temps et matériau !}
    \end{block}
\end{frame}

% ==============================================================================
% SECTION 7: BILAN (2 min)
% ==============================================================================
\section{Bilan}

\begin{frame}{État Final du Projet}
    \begin{columns}
        \begin{column}{0.5\textwidth}
            \textbf{Stack Technique :}
            \begin{table}[]
                \footnotesize
                \rowcolors{2}{gray!10}{white}
                \begin{tabular}{l | l}
                    \toprule
                    \textbf{Module} & \textbf{Technologie} \\
                    \midrule
                    App Mobile & Android (Java) \\
                    Backend & Python FastAPI \\
                    Hardware & Pi 5 / lgpio \\
                    Mécanique & PLA / SolidPython \\
                    \bottomrule
                \end{tabular}
            \end{table}
        \end{column}
        \begin{column}{0.45\textwidth}
            \textbf{Prochaines Étapes :}
            \begin{itemize}
                \item Finaliser impression 3D
                \item Concevoir support périph.
                \item Enquêter blocage UART
            \end{itemize}
        \end{column}
    \end{columns}
    
    \vspace{0.3cm}
    \begin{block}{Retour d'Expérience}
        Projet complet : Hardware + Software + Mobile + Fabrication.
    \end{block}
\end{frame}

\begin{frame}{Démonstration : Captures d'Écran}
    \begin{columns}
        \begin{column}{0.45\textwidth}
            \centering
            \includegraphics[width=0.9\textwidth]{pic/app_connex_login_dash.png}
            \par\vspace{0.2cm}\small \textbf{Connexion/Inscription}\\Login automatiquement
        \end{column}
        % \begin{column}{0.32\textwidth}
        %     \centering
        %     \includegraphics[width=0.5\textwidth]{pic/app_dashboard.png}
        %     \par\vspace{0.2cm}\small \textbf{Dashboard}\\Accès utilisateur
        % \end{column}
        \begin{column}{0.45\textwidth}
            \centering
            \includegraphics[width=0.9\textwidth]{pic/app_admin_3.png}
            \par\vspace{0.2cm}\small \textbf{Admin}\\Gestion canaux
        \end{column}
    \end{columns}
    
    \vspace{0.3cm}
    \begin{block}{Interface Vivid Palette}
        \begin{itemize}
            \item \textcolor[HTML]{2ECC71}{$\blacksquare$} Vert = Prêt \quad
            \textcolor[HTML]{F1C40F}{$\blacksquare$} Jaune = Sélection \quad
            \textcolor[HTML]{E57373}{$\blacksquare$} Rouge = Danger
        \end{itemize}
    \end{block}
\end{frame}

\begin{frame}[fragile]{Réflexion : Utilisation de l'IA}
    \begin{columns}
        \begin{column}{0.48\textwidth}
            \textbf{Outils Utilisés :}
            \begin{itemize}
                \item \textbf{Gemini} (principal)
                \item Codex (GPT)
                \item Copilot (Claude Opus)
            \end{itemize}
            
            \vspace{0.2cm}
            \textbf{Avantages :}
            \begin{itemize}
                \item Productivité accrue
                \item Prototypage rapide
            \end{itemize}
        \end{column}
        \begin{column}{0.48\textwidth}
            \textbf{Limites Constatées :}
            \begin{itemize}
                \item Bibliothèques obsolètes
                \item Solutions inutilement complexes
                \item Bugs subtils à relire
            \end{itemize}
            
            \vspace{0.2cm}
            \textbf{Mon Apprentissage :}
            \begin{itemize}
                \item Relecture du code
                \item Compréhension de l'architecture
            \end{itemize}
        \end{column}
    \end{columns}
\end{frame}

\begin{frame}[fragile]{Exemple de Bug IA : Timestamps Incohérents}
    \begin{columns}
        \begin{column}{0.5\textwidth}
            \textbf{Code IA Original :}
            \begin{lstlisting}[language=Python, basicstyle=\scriptsize\ttfamily]
    last_activity_time = time.time()
    session_start_time = time.time()
    last_clock_update = time.time()
    system_state = "ACTIVE"
            \end{lstlisting}
            \vspace{0.2cm}
            \textcolor{red}{$\times$ 3 appels $\rightarrow$ 3 timestamps différents}
        \end{column}
        \begin{column}{0.48\textwidth}
            \textbf{Code Corrigé :}
            \begin{lstlisting}[language=Python, basicstyle=\scriptsize\ttfamily]
    now = time.time()
    last_activity_time = now
    session_start_time = now
    last_clock_update = now
    system_state = "ACTIVE"
            \end{lstlisting}
            \vspace{0.2cm}
            \textcolor{green!60!black}{$\checkmark$ 1 appel $\rightarrow$ timestamps synchronisés}
        \end{column}
    \end{columns}
    
    \vspace{0.3cm}
    \begin{block}{Résultat}
        Avant : Affichage secondes saccadé (0.1s - 2s irrégulier). \\
        Après : Mise à jour fluide et régulière.
    \end{block}
\end{frame}

\begin{frame}{Conclusion}
    DistCapsule a évolué d'un simple concept vers un \textbf{prototype fonctionnel} de distributeur connecté, basé sur un \textbf{système embarqué} (Raspberry Pi 5).
    \vspace{0.5cm}
    
    \textbf{Points forts :}
    \begin{itemize}
        \item Expérience utilisateur fluide (App native, biométrie, parcours simple).
        \item Architecture logicielle découplée (3-tiers, FastAPI, Core Agent multi-thread).
        \item Respect de la vie privée (traitement et stockage 100\% local).
    \end{itemize}
\end{frame}

\begin{frame}{Merci}
    \centering
    \includegraphics[width=0.9\textwidth,height=0.65\textheight,keepaspectratio]{pic/DistCapsule.jpg}
    
    \vspace{0.5cm}
    \Large \textbf{Place à la démonstration !}
\end{frame}

\end{document}
