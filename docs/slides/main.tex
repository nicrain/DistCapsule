\DocumentMetadata{
  lang=fr-FR,
  pdfstandard=ua-1,
  testphase={phase-I,table,title}
}
\documentclass[xcolor=table]{beamer}
\usepackage{graphicx}

% --- Core French Language Support ---
\usepackage[T1]{fontenc}
\usepackage[utf8]{inputenc}
\usepackage[french]{babel}
\usepackage{listings}
\usepackage{booktabs} % For nicer tables
\usepackage{caption} % Fix for captionof if needed

\usetheme{Darmstadt}
\usecolortheme{dolphin}
\usefonttheme{professionalfonts}

\usepackage{csquotes}
\usepackage[style=numeric, backend=biber]{biblatex}
\addbibresource{ref.bib}

\DefineBibliographyStrings{french}{
  bibliography = {R{\'e}f{\'e}rences},
}

\AtBeginSection[]{
\begin{frame}
    \tableofcontents[sectionstyle=show/shaded,subsectionstyle=show/shaded/hide]
\end{frame}
}

%% --- Title Page Metadata ---
\title[DistCapsule]{DistCapsule : Distributeur de Capsules}
\subtitle{Projet AAA (Android, Arduino, Agent Artificiel)}
\author[Z. Wang \& X. Tang]{Zhaoyu WANG \& Xinqi TANG}
\date[\today]{\today}
\institute[M2 MIASHS]{Université Paris 8 \\ Master Technologies et Handicaps}

%% --- Logo Configuration ---
\titlegraphic{
   \includegraphics[width=1.5cm]{pic/logo-handi.png}
   \hspace{0.5cm}
   \includegraphics[width=1.5cm]{pic/logo-p8.png}
}
\logo{\includegraphics[width=1cm]{pic/logo-p8.png}}

%% --- Code Snippet Style ---
\lstset{
    basicstyle=\ttfamily\scriptsize, % Smaller font for code fitting
    keywordstyle=\color{blue},
    stringstyle=\color{green!60!black},
    commentstyle=\color{gray},
    numbers=left,
    numberstyle=\tiny\color{gray},
    frame=single,
    breaklines=true,
    language=Python
}

\begin{document}

{
\logo{}
\begin{frame}
    \titlepage   
\end{frame}
}

% --- Section 1: Introduction ---
\section{Introduction}
\begin{frame}{Pr{\'e}sentation du Projet}
    \begin{block}{Objectif}
        Concevoir un syst{\`e}me de distribution s{\'e}curis{\'e} et personnalis{\'e}, transformant un simple pr{\'e}sentoir en une \textbf{"bo\^ite aux lettres intelligentes"} accessible par biom{\'e}trie.
    \end{block}

    \begin{itemize}
        \item \textbf{Cible} : Gestion de capsules de caf{\'e} (bureau/domicile).
        \item \textbf{Innovation} : Authentification double (Visage + Empreinte) et architecture autonome.
        \item \textbf{État Actuel} : Phase S6 (Intégration Mobile \& API).
    \end{itemize}
\end{frame}

% --- Section 2: Architecture AAA ---
\section{Architecture AAA}
\subsection{Concept Global}
\begin{frame}{Le Mod{\`e}le AAA}
    Le projet repose sur une architecture tripartite classique en IoT :
    \begin{enumerate}
        \item \textbf{Android} (Interface Humaine)
        \item \textbf{Arduino / Raspberry Pi} (Interface Physique)
        \item \textbf{Agent Artificiel} (Cerveau D{\'e}cisionnel)
    \end{enumerate}
\end{frame}

\subsection{Détails des Couches}
\begin{frame}{1. Android \& 2. Hardware (Arduino/Pi)}
    \begin{columns}
        \begin{column}{0.48\textwidth}
            \begin{alertblock}{Android (App)}
                \begin{itemize}
                    \item Interface de gestion (Admin).
                    \item Enr{\^o}lement des utilisateurs.
                    \item Consultation des logs via API REST.
                    \item Notifications d'{\'e}tat.
                \end{itemize}
            \end{alertblock}
        \end{column}
        \begin{column}{0.48\textwidth}
            \begin{exampleblock}{Hardware (Pi 5)}
                \begin{itemize}
                    \item \textbf{Action} : 5x Servomoteurs (lgpio).
                    \item \textbf{Vision} : Cam{\'e}ra Pi Module 3.
                    \item \textbf{Toucher} : Capteur Empreinte (UART).
                    \item \textbf{Affichage} : \'{E}cran IPS (SPI).
                \end{itemize}
            \end{exampleblock}
        \end{column}
    \end{columns}
\end{frame}

\begin{frame}{3. Agent Artificiel (Le Cœur)}
    L'Agent est un programme Python autonome qui orchestre tout :
    \begin{itemize}
        \item \textbf{Multi-threading} : S{\'e}paration des t{\^a}ches lourdes (IA) et de l'UI.
        \item \textbf{Machine \`a {\'{E}}tats} : Gestion des modes (Veille, Scan, D{\'e}verrouill{\'e}).
        \item \textbf{D{\'e}cision} : "Est-ce un utilisateur connu ? A-t-il le droit d'ouvrir cette bo\^ite ?"
    \end{itemize}
\end{frame}

% --- Section 3: Application Mobile ---
\section{Application Mobile}
\subsection{Rôles \& Fonctionnalités}
\begin{frame}{Expérience Utilisateur (UX)}
    L'application s'adapte dynamiquement au rôle de l'utilisateur :
    \begin{columns}
        \begin{column}{0.48\textwidth}
            \begin{block}{Utilisateur Standard}
                \begin{itemize}
                    \item \textbf{One-Tap Dispense} : Bouton "Mon Café" (Ouvre le canal assigné).
                    \item \textbf{État} : Visualisation des statuts biométriques (Visage/Empreinte).
                \end{itemize}
            \end{block}
        \end{column}
        \begin{column}{0.48\textwidth}
            \begin{alertblock}{Administrateur}
                \begin{itemize}
                    \item \textbf{Gestion} : Créer/Supprimer utilisateurs.
                    \item \textbf{Maintenance} : Contrôle manuel des 5 canaux.
                    \item \textbf{Enrôlement} : Déclenchement à distance de l'enregistrement biométrique.
                \end{itemize}
            \end{alertblock}
        \end{column}
    \end{columns}
\end{frame}

\subsection{Architecture Technique}
\begin{frame}{Communication App $\leftrightarrow$ Pi}
    \begin{itemize}
        \item \textbf{Protocole} : REST API (FastAPI).
        \item \textbf{Sécurité} : Authentification par \textbf{Token} (UUID) lié au dispositif.
        \item \textbf{Flux d'Enrôlement} :
        \begin{enumerate}
            \item Admin clique "Enrôler Visage" sur l'App.
            \item App envoie \texttt{POST /command/enroll\_face}.
            \item Pi se réveille, active la caméra et guide l'utilisateur.
            \item Résultat mis à jour en base de données.
        \end{enumerate}
    \end{itemize}
\end{frame}

% --- Section 4: Réalisation Physique ---
\section{Réalisation Physique}
\subsection{Montage Matériel}
\begin{frame}{Composants et Assemblage}
    Le système intègre divers modules pilotés par le Raspberry Pi 5 :
    \begin{itemize}
        \item \textbf{Actionneurs} : 5x Servomoteurs SG90 pour la distribution.
        \item \textbf{Capteurs} : Caméra Module 3 et Lecteur d'empreintes DY-50.
        \item \textbf{Interface} : Écran LCD IPS et bouton de réveil physique.
    \end{itemize}
    \begin{alertblock}{Gestion de l'énergie}
        Utilisation d'une alimentation externe 5V dédiée pour les moteurs afin de protéger le Pi 5.
    \end{alertblock}
\end{frame}

\subsection{Conception 3D}
\begin{frame}{Modélisation Paramétrique}
    Le châssis a été conçu via une approche \textbf{"Code-to-CAD"} :
    \begin{itemize}
        \item \textbf{Outils} : OpenSCAD \& SolidPython.
        \item \textbf{Avantage} : Design paramétrique permettant des ajustements rapides des dimensions.
    \end{itemize}
    
    \begin{block}{Éléments imprimés}
        \begin{itemize}
            \item \textbf{Base\_Distributeur} : Structure principale de stockage.
            \item \textbf{Control\_Box} : Boîtier de protection pour l'électronique.
            \item \textbf{Pusher \& Drawer} : Mécanisme d'éjection des capsules.
            \item \textbf{Servo\_Arm} : Extensions sur mesure pour les palonniers.
        \end{itemize}
    \end{block}
\end{frame}

\begin{frame}{Itération du Design}
    \begin{columns}
        \begin{column}{0.6\textwidth}
            \begin{itemize}
                \item \textbf{1. Inspiration} : Produit commercial (Amazon) à chargement vertical.
                \item \textbf{2. Adaptation (V1)} : Reproduction imprimable (Conseil de D. Archambault).
                \item \textbf{3. Automatisation (V2)} : Évolution vers une distribution gravitaire par le bas (\texttt{distCaps}).
            \end{itemize}
            
            \begin{alertblock}{État Actuel (Prototype Hybride)}
                Contrainte temporelle $\rightarrow$ Solution mixte :
                \begin{itemize}
                    \item \textbf{1 Canal} : Nouveau design auto (Pusher).
                    \item \textbf{4 Canaux} : Design manuel vertical.
                \end{itemize}
            \end{alertblock}
        \end{column}
        \begin{column}{0.38\textwidth}
            \begin{figure}
                \centering
                \includegraphics[width=0.9\textwidth]{pic/amazon_ref.jpg}
                \caption{Inspiration Originale}
            \end{figure}
        \end{column}
    \end{columns}
\end{frame}

% --- Section 4: D{\'e}fis Techniques ---
\section{D{\'e}fis Techniques}
\subsection{Vision par Ordinateur}
\begin{frame}[fragile]{Adaptation Mat{\'e}rielle : Rotation}
    La cam{\'e}ra est mont{\'e}e physiquement \`a 90°. L'algorithme doit corriger l'image avant l'analyse IA.
    
    \begin{lstlisting}[caption={Correction de l'orientation (Python/OpenCV)}]
# Dans hardware/face_system.py
def scan(self):
    ret, frame = self.cap.read()
    if not ret: return None

    # Rotation de 90 degres (Clockwise)
    frame = cv2.rotate(frame, cv2.ROTATE_90_CLOCKWISE)

    # Amelioration (CLAHE) et Detection...
    # ...
    \end{lstlisting}
\end{frame}

\subsection{Concurrence}
\begin{frame}{Architecture Multi-thread}
    Pour {\'e}viter que la reconnaissanc{\`e} faciale (lente) ne bloque le compte \`a rebours \`a l'{\'e}cran (rapide) :
    
    \begin{table}[]
        \centering
        \rowcolors{2}{gray!10}{white}
        \begin{tabular}{l | l | l}
            \toprule
            \textbf{Thread} & \textbf{R{\^o}le} & \textbf{Priorit{\'e}} \\
            \midrule
            Main Loop & Gestion UI (ontiti), Boutons & Haute (Temps r{\'e}el) \\
            Face Worker & Capture \& Analyse IA & Basse (Arri{\`e}re-plan) \\
            API Server & R{\'e}pondre aux requ\^{e}tes HTTP & Moyenne \\
            \bottomrule
        \end{tabular}
        \caption{R{\'e}partition des t{\^a}ches}
    \end{table}
\end{frame}

% --- Section 5: Conclusion ---
\section{Conclusion}
\begin{frame}{Perspectives \& Améliorations}
    \begin{itemize}
        \item \textbf{Prochaines {\'{e}}tapes (S7)} :
        \begin{itemize}
            \item Finalisation de l'application Android (Kotlin).
            \item Bo\^itier imprim{\'e} en 3D complet.
        \end{itemize}
        \item \textbf{Am{\'e}liorations futures} :
        \begin{itemize}
            \item Support MQTT pour le contr{\^o}le hors r{\'e}seau local.
            \item Analyse d'inventaire (compter les capsules restantes par vision).
        \end{itemize}
    \end{itemize}
\end{frame}

\begin{frame}[allowframebreaks]{R{\'e}f{\'e}rences}
     \printbibliography[heading=none] 
\end{frame}

\end{document}
