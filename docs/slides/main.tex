\DocumentMetadata{
  lang=fr-FR,
  pdfstandard=ua-1,
  testphase={phase-I,table,title}
}
\documentclass[xcolor=table, aspectratio=169]{beamer}
\usepackage{graphicx}

% --- Core French Language Support ---
\usepackage[T1]{fontenc}
\usepackage[utf8]{inputenc}
\usepackage[french]{babel}
\usepackage{listings}
\usepackage{booktabs}
\usepackage{caption}
\usepackage{subcaption}
\usepackage{tikz} 

\usetheme{Madrid}
\usecolortheme{dolphin}
\usefonttheme{professionalfonts}

\usepackage{csquotes}
\usepackage[style=numeric, backend=biber]{biblatex}
\addbibresource{ref.bib}
\DefineBibliographyStrings{french}{bibliography = {R\'ef\'erences}}

\AtBeginSection[]{
\begin{frame}
    \frametitle{Sommaire}
    \tableofcontents[currentsection]
\end{frame}
}

% --- Code Style ---
\lstset{
    basicstyle=\ttfamily\scriptsize,
    keywordstyle=\color{blue!80!black}\bfseries,
    stringstyle=\color{green!60!black},
    commentstyle=\color{gray}\itshape,
    breaklines=true,
    language=Python,
    frame=single
}

%% --- Title ---
\title[DistCapsule]{DistCapsule : Distributeur Connect\'e}
\subtitle{Syst\`eme IoT Biom\'etrique sur Raspberry Pi 5}
\author[Z. Wang \& X. Tang]{Zhaoyu WANG \& Xinqi TANG}
\date[\today]{\today}
\institute[M2 MIASHS]{Universit\'e Paris 8 \\ Master Technologies et Handicaps}

\titlegraphic{
   \includegraphics[height=1.2cm]{pic/logo-handi.png} \hfill
   \includegraphics[height=1.2cm]{pic/logo-p8.png}
}
\logo{\includegraphics[width=0.8cm]{pic/logo-p8.png}}

\begin{document}

{
\setbeamertemplate{logo}{}
\begin{frame}
    \titlepage
\end{frame}
}

% ==============================================================================
% SECTION 1: CONTEXTE
% ==============================================================================
\section{Contexte \& Objectifs}
\begin{frame}{Le Besoin}
    \begin{columns}
        \begin{column}{0.6\textwidth}
            \begin{block}{Le Constat}
                Dans un espace café partagé, les capsules sont souvent mélangées en vrac ou stockées individuellement, encombrant inutilement le plan de travail.
            \end{block}
            \vspace{0.5cm}
            \begin{itemize}
                \item \textbf{Problème} : Désordre visuel, difficulté à identifier ses propres capsules, perte d'espace.
                \item \textbf{Solution} : Un stockage vertical, centralisé et personnalisé qui libère de l'espace.
            \end{itemize}
        \end{column}
        \begin{column}{0.38\textwidth}
            \begin{figure}
                \includegraphics[width=\textwidth,height=4cm,keepaspectratio]{pic/amazon_ref.jpg}
                \caption{Inspiration}
            \end{figure}
        \end{column}
    \end{columns}
\end{frame}

% ==============================================================================
% SECTION 2: ARCHITECTURE & FLUX
% ==============================================================================
\section{Architecture Globale}

\begin{frame}{Topologie du Syst\`eme}
    \begin{block}{Architecture 3-Tiers}
        Le syst\`eme est divis\'e en trois couches ind\'ependantes communiquant via HTTP et SQLite.
    \end{block}
    \vspace{0.3cm}
    \begin{figure}
        \centering
        % TODO: Ins\'erer un sch\'ema global (App -> API -> Hardware)
        % Nom du fichier: pic/system_architecture.png
        \includegraphics[width=0.9\textwidth,height=4cm,keepaspectratio]{pic/system_architecture.png}
        \caption{Flux de donn\'ees : Du smartphone au servomoteur}
    \end{figure}
\end{frame}

\begin{frame}{S\'equence d\'Action (Workflow)}
    Que se passe-t-il quand l'utilisateur clique sur "Mon Caf\'e" ?
    \vspace{0.2cm}
    \begin{enumerate}
        \item \textbf{App} : Envoie \texttt{POST /command/unlock} avec le Token.
        \item \textbf{API} : Vérifie le Token et ajoute la commande en BDD (\texttt{Pending\_Commands}).
        \item \textbf{Core (Main Loop)} :
        \begin{itemize}
            \item D\'etecte la nouvelle commande.
            \item Active le servomoteur correspondant (via \texttt{lgpio}).
            \item Affiche un compte \`a rebours sur l'\'ecran IPS.
        \end{itemize}
        \item \textbf{App} : Reçoit la confirmation et vibre.
    \end{enumerate}
\end{frame}

\begin{frame}{Réseau : Stabilité \& Connectivité}
    \begin{block}{Le Défi}
        Les smartphones modernes refusent souvent les connexions Wi-Fi "sans internet" (No Gateway).
    \end{block}
    
    \begin{exampleblock}{Solution Finale (V1.1)}
        \begin{itemize}
            \item \textbf{Mode} : Hotspot Standard avec passerelle DHCP activée.
            \item \textbf{Avantage} : Connexion immédiate et stable sur 100\% des appareils Android/iOS testés.
            \item \textbf{UX} : Bouton "Connecter au Wi-Fi" intégré directement dans l'application.
        \end{itemize}
    \end{exampleblock}
\end{frame}

% ==============================================================================
% SECTION 3: MOBILE & UX
% ==============================================================================
\section{Exp\'erience Mobile}

\begin{frame}{Interface Utilisateur (Vivid Palette)}
    \begin{columns}
        \begin{column}{0.5\textwidth}
            Une interface native Android (Java) conçue pour la clarté :
            \begin{itemize}
                \item \textbf{Vert} : Tout est pr\^et.
                \item \textbf{Jaune} : Action requise (Wi-Fi).
                \item \textbf{Rouge} : Danger (Suppression).
            \end{itemize}
        \end{column}
        \begin{column}{0.45\textwidth}
            \centering
            % TODO: Mettre les vrais screenshots
            \includegraphics[width=0.45\textwidth]{pic/app_login.png} \hfill
            \includegraphics[width=0.45\textwidth]{pic/app_dashboard.png}
            \captionof{figure}{Login \& Dashboard}
        \end{column}
    \end{columns}
\end{frame}

% ==============================================================================
% SECTION 4: HARDWARE & \'ELECTRONIQUE
% ==============================================================================
\section{C\^ablage \& \'Electronique}

\begin{frame}{Sch\'ema de C\^ablage (Wiring)}
    Le Raspberry Pi 5 est le chef d'orchestre, pilotant plusieurs p\'eriph\'eriques simultan\'es.
    
    \begin{figure}
        \centering
        % TODO: Ins\'erer le sch\'ema de c^ablage (Fritzing ou Photo annot\'ee)
        % Nom: pic/hardware_wiring.png
        \includegraphics[width=0.8\textwidth,height=4.5cm,keepaspectratio]{pic/hardware_wiring.png}
        \caption{Connexions : GPIO (PWM), UART (Fingerprint), SPI (LCD)}
    \end{figure}
    
    \begin{itemize}
        \item \textbf{Servos} : GPIO 12, 13, 18, 19, 23 (Soft-PWM).
        \item \textbf{S\'ecurit\'e} : Alimentation 5V externe s\'epar\'ee pour isoler le Pi du bruit moteur.
    \end{itemize}
\end{frame}

\begin{frame}{Interface Embarqu\'ee (\'Ecran IPS)}
    L'\'ecran ST7789 (240x240) fournit un feedback visuel imm\'ediat \`a l'utilisateur devant la machine.
    
    \begin{figure}
        \centering
        % TODO: Ins\'erer une matrice de photos de l'\'ecran (Veille, Compte \`a rebours, Erreur)
        % Nom: pic/ips_ui_matrix.png
        \includegraphics[width=0.9\textwidth,height=3.5cm,keepaspectratio]{pic/ips_ui_matrix.png}
        \caption{\'Etats de l'\'ecran : Veille, Scan, D\'ecompte, Erreur}
    \end{figure}
\end{frame}

% ==============================================================================
% SECTION 5: D\'EVIS & 3D
% ==============================================================================
\section{D\'efis Techniques}

\begin{frame}{Le D\'efi du Pi 5 : GPIO \& OS}
    \begin{alertblock}{Probl\`eme}
        Le Raspberry Pi 5 utilise une nouvelle puce RP1, rendant \texttt{RPi.GPIO} obsol\`ete.
    \end{alertblock}
    
    \begin{block}{Solution : lgpio}
        Réécriture complète de la couche driver (\texttt{servo\_control.py}) pour utiliser \texttt{lgpio}, permettant un contrôle PWM logiciel stable sur 5 canaux simultanés.
    \end{block}
\end{frame}

\begin{frame}{Mod\'elisation 3D Param\'etrique}
    Approche "Code-to-CAD" avec OpenSCAD :
    \begin{columns}
        \begin{column}{0.6\textwidth}
            \begin{itemize}
                \item Design modulaire (Base, Tiroir, Pousseur).
                \item Ajustement rapide des tol\'erances pour l'impression 3D.
            \end{itemize}
        \end{column}
        \begin{column}{0.35\textwidth}
            \centering
            % TODO: Rendu 3D ou photo de la pi\`ece
            \includegraphics[width=\textwidth]{pic/3d_render.png} 
        \end{column}
    \end{columns}
\end{frame}

% ==============================================================================
% SECTION 6: CONCLUSION
% ==============================================================================
\section{Bilan}

\begin{frame}{\'Etat Final du Projet}
    \begin{table}[]
        \centering
        \rowcolors{2}{gray!10}{white}
        \begin{tabular}{l | l | l}
            \toprule
            \textbf{Module} & \textbf{Techno} & \textbf{Statut} \\
            \midrule
            App Android & Java / Retrofit & \textcolor{green!60!black}{\textbf{V1.1 Prod}} \\
            API Server & Python FastAPI & \textcolor{green!60!black}{\textbf{Stable}} \\
            Hardware & Pi 5 / lgpio & \textcolor{green!60!black}{\textbf{Fonctionnel}} \\
            M\'ecanique & PLA / OpenSCAD & \textcolor{orange}{\textbf{Assemblage}} \\
            \bottomrule
        \end{tabular}
    \end{table}
\end{frame}

\begin{frame}{Conclusion}
    \textbf{DistCapsule} est une preuve de concept r\'eussie d'un IoT complet, s\'ecuris\'e et convivial.
    \vspace{1cm}
    \begin{center}
        \huge \textbf{D\'emonstration ?}
    \end{center}
\end{frame}

\begin{frame}[allowframebreaks]{R\'ef\'erences}
     \printbibliography[heading=none] 
\end{frame}

\end{document}